\documentclass[11pt,a4paper]{article}
\usepackage[utf8]{inputenc}
\usepackage[T1]{fontenc}
\usepackage[polish]{babel}
\usepackage{amsmath}
\usepackage{amsfonts}
\usepackage{amssymb}
\author{Paweł Cichowski}
\date{}
\title{SBD Projekt 1}
\begin{document}

\maketitle

\section{Wprowadzenie}

Celem projektu było zaimplementowanie programu sortującego plik metodą sortowania polifazowego Fibonacciego. Odczyty oraz zapisy danych symulowane są przez blokowe operacje dyskowe. Zadanie zostało zrealizowane w języku C.

Program jest podzielony na dwie warstwy abstrakcji: warstwę sortowania oraz warstwę odczytu i zapisu, która udostępnia dwie funkcje: pobrania kolejnego rekordu z pliku oraz dopisania rekordu do pliku.

Wylosowanym typem rekordu był numer dowodu osobistego - obywatel, uporządkowane po numerze dowodu. W programie przyjęto, że klucze rekordów są unikalne.

\section{Instrukcja obsługi}

\section{Wyniki eksperymentu}

\subsection{Opis eksperymentu}

Przeprowadzony został eksperyment, polegający na wygenerowaniu wybranej liczby losowych rekordów oraz przeanalizowanie liczby serii. Następnie plik został posortowany metodą sortowania polifazowego oraz zapisana została liczba faz potrzebnych na zrealizowanie sortowania.

\subsection{Dane wejściowe}

\begin{tabular}{|c|c|c|}
\hline 
Nr. eksperymentu & Liczba rekordów & Liczba serii \\ 
\hline 
1 & 10 & 6 \\ 
\hline 
2 & 100 & 52 \\ 
\hline 
3 & 1000 & 512 \\
\hline
4 & 10.000 & 5031 \\
\hline 
5 & 100.000 & • \\ 
\hline 
6 & 1.000.000 & • \\ 
\hline 
7 & 10.000.000 & • \\ 
\hline 
\end{tabular} 



\subsection{Rezultaty sortowania}


\end{document}