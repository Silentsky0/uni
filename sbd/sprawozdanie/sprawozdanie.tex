\documentclass[11pt,a4paper]{article}
\usepackage[utf8]{inputenc}
\usepackage[T1]{fontenc}
\usepackage[polish]{babel}
\usepackage{amsmath}
\usepackage{graphicx}
\usepackage{amsfonts}
\usepackage{amssymb}
\author{Paweł Cichowski}
\date{}
\title{SBD Projekt 1}
\begin{document}

\maketitle

\section{Wprowadzenie}

Celem projektu było zaimplementowanie programu sortującego plik metodą sortowania polifazowego Fibonacciego. Odczyty oraz zapisy danych symulowane są przez blokowe operacje dyskowe. Zadanie zostało zrealizowane w języku C.

Program jest podzielony na dwie warstwy abstrakcji: warstwę sortowania oraz warstwę odczytu i zapisu, która udostępnia dwie funkcje: pobrania kolejnego rekordu z pliku oraz dopisania rekordu do pliku.

Wylosowanym typem rekordu był numer dowodu osobistego - obywatel, uporządkowane po numerze dowodu. W programie przyjęto, że klucze rekordów są unikalne.

\section{Instrukcja obsługi programu}

Program obsługuje kilka parametrów pozycyjnych, takich jak:
\begin{itemize}
\item -h, -{}-help - wyświetla dostępne komendy
\item -n, -{}-n-to-generate - umożliwia ustawienie liczby rekordów do losowego wygenerowania
\item -c, -{}-print-contents - wyświetla rekordy pliku wejściowego i posortowanego
\item -p, -{}-print-every-phase - wyświetla zawartość taśm po każdej fazie sortowania
\item -k, -{}-keyboard - umożliwia manualne wprowadzenie danych z klawiatury
\item -f, -{}-file-path - specyfikuje ścieżkę, z której rekordy zostaną pobrane
\end{itemize}

\section{Wyniki eksperymentu}

\subsection{Opis eksperymentu}

Przeprowadzony został eksperyment, polegający na wygenerowaniu wybranej liczby losowych rekordów oraz przeanalizowanie liczby serii. Następnie plik został posortowany metodą sortowania polifazowego oraz zapisana została liczba faz potrzebnych na zrealizowanie sortowania.

\subsection{Dane wejściowe}

\begin{tabular}{|c|c|c|}
\hline 
Nr. eksperymentu & Liczba rekordów & Liczba serii \\ 
\hline 
1 & 10 & 6 \\ 
\hline 
2 & 100 & 52 \\ 
\hline 
3 & 1000 & 512 \\
\hline
4 & 10.000 & 5031 \\
\hline 
5 & 100.000 & 49.197 \\ 
\hline 
6 & 1.000.000 & 491.366 \\ 
\hline 
\end{tabular} 

Liczba serii jest zgodna z teorią, czyli w przyblizeniu połowie liczby rekordów.

\subsection{Rezultaty sortowania}

\begin{tabular}{|p{0.1\linewidth}|p{0.2\linewidth}|p{0.2\linewidth}|p{0.2\linewidth}|p{0.2\linewidth}|p{0.2\linewidth}|}
\hline 
Numer eksperymentu & Liczba rekordów & Liczba faz sortowania & Liczba operacji dyskowych & Teoretyczna liczba faz & Teoretyczna liczba operacji dyskowych \\ 
\hline 
1 & 10 & 3 & 21 & 4 & 13 \\ 
\hline 
2 & 100 & 8 & 260 & 9 & 231 \\ 
\hline 
3 & 1000 & 13 & 3676 & 13 & 3453 \\ 
\hline 
4 & 10.000 & 18 & 48670 & 18 & 45962 \\ 
\hline 
5 & 100.000 & 23 & 740377 & 23 & 573658 \\ 
\hline 
6 & 1.000.000 & 27 & 8399027 & 27 & 6887565 \\ 
\hline 
\end{tabular} 

\includegraphics[scale=0.7]{../../../images/Screenshots/Screenshot from 2022-12-04 23-43-52.png} 

\includegraphics[scale=0.7]{../../../images/Screenshots/Screenshot from 2022-12-04 23-43-13.png} 

Na podstawie zebranych danych oraz utworzonych wykresów można stwierdzić, że liczba faz sortowania zbiega do wartości teoretycznej dla coraz większej ilości rekordów. 

Liczba operacji dyskowych w pewnym stopniu odbiega od wartości teoretycznych, wnioskiem może być to, że program wykonuje dodatkowe, redundantne operacje dyskowe podczas sortowania.

\end{document}